\anonsection{Заключение}

% В заключение данной работы хочется отметить, что основные поставленные цели были достигнуты, была создана модель простой ячейки логистики, и рассмотрены разные частные случаи дорог. После этого они были использованы в большой модели и протестированы. В будущем можно попытаться  рассмотреть больше частных случаев и исследовать более сложные формулы соединения и разделения потоков не только на 2, но и на большее число потоков. Также можно будет рассмотреть использование светофоров и других средств, регулирующих движение транспорта..

На первом этапе работы были проанализированы научные статьи по теме моделирования логистики с использованием методов системной динамики. В результате анализа были выбраны 3 статьи \cite{benaich2015exploring,jin2022variable,crainic2009models}, описывающие модели участка дорог, а также рассматривающие возможные оптимизации транспортных систем.

На втором этапе была создана модель простой ячейки прямой дороги. Также была протестирована модель, состоящая из двух последовательных простых ячеек.

На третьем этапе были созданы более сложные варианты ячеек, выполняющие соединение и разделение двух потоков машин, согласно заданным коэффициентам приоритета.

На четвёртом этапе работы для тестирования применимости разработанных модулей была создана сложная модель, описывающая протяжённый участок дороги с разделением и слиянием потоков машин.

В будущих работах можно смоделировать более сложные варианты конфигурации дорожных соединений, а также исследовать модели светофоров и других средств регулирования трафика.

\clearpage