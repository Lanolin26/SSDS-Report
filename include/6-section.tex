\section{Тестирование}

\subsection{Описание модели}

На рисунке \ref{img:sec6_model} представлена получившаяся модель системной динамики прямого соединения двух ячеек.

Модель представляет собой шесть стоков (ячейки) и восемь потоков: один входящий, шесть промежуточных и один выходящий.
Так же присутствует ряд переменных, которые задают поведение ячеек (каждый параметр индивидуален для своей ячейки):
\begin{itemize}
    \item[Speed] - ограничение скорости машины в ячейке;
    \item[V0] - входящая скорость машины;
    \item[N0] - количество машины на 1 км, которые могут находиться в ячейке;
    \item[Ln] - длинна участка ячейки;
    \item[C0] - пропускная способность ячейки на 1 км;
    \item[s 0] - количество машин, которые могут въехать с предыдущей ячейки;
    \item[r 0] - количество машин, которое может выехать из ячейки и попасть в следующую;
\end{itemize}

\addimghere{sec6/model.png}{1}{Модель одной ячейки дороги}{img:sec6_model}

\subsection{Описание результатов}

На рисунке \ref{img:sec6_res1} показано, что при заданных начальных параметрах, указанных слева, .

\addimghere{sec6/res1.png}{1}{Результаты запуска модели}{img:sec6_res1}

На рисунке \ref{img:sec6_res2} показано, что при заданных начальных параметрах, указанных слева, .

\addimghere{sec6/res2.png}{1}{Результаты запуска модели}{img:sec6_res2}

На рисунке \ref{img:sec6_res3} показано, что при заданных начальных параметрах, указанных слева, 

\addimghere{sec6/res3.png}{1}{Результаты запуска модели}{img:sec6_res3}

На рисунке \ref{img:sec6_res4} показано, что при заданных начальных параметрах, указанных слева, 

\addimghere{sec6/res4.png}{1}{Результаты запуска модели}{img:sec6_res4}

\clearpage