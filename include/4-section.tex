\section{Модель разделения дорог}

\subsection{Описание модели}

На рисунке \ref{img:sec4_model} представлена получившаяся модель системной динамики разделения одной ячейки на две. Листинг модели в формате xmile представлен в приложении \hyperlink{app-c}{В}.

Модель представляет собой три стока (ячейки) и пять потоков: входящий, два промежуточных и два выходящих.
Так же присутствует ряд переменных, которые задают поведение ячеек (каждый параметр индивидуален для своей ячейки):
\begin{itemize}
    \item[Speed] - ограничение скорости машины в ячейке;
    \item[V0] - входящая скорость машины;
    \item[N0] - количество машины на 1 км, которые могут находиться в ячейке;
    \item[Ln] - длинна участка ячейки;
    \item[C0] - пропускная способность ячейки на 1 км;
    \item[s 0] - количество машин, которые могут въехать с предыдущей ячейки;
    \item[r 0] - количество машин, которое может выехать из ячейки и попасть в следующую;
\end{itemize}

\addimghere{sec4/model.png}{1}{Модель разделения одной ячейки на две}{img:sec4_model}

\subsection{Описание результатов}

На рисунке \ref{img:sec4_res1} показано, что при заданных начальных параметрах, указанных слева, мы можем видеть, что в начале пропускная способность поток максимальна. Однако после заполнения ячейки С, машины которые должны ехать в эту ячейку остаются в ячейке А. До примерно 25 часов по графику происходит наполнение ячеек. С учетом, что коэффициент деления машин равняется $0.32$ на ячейку B, то основной поток машин идет на ячейку C. Однако, начальные параметры подобраны таким образом, что ячейка В успевает быстрее опустошаться, нежели ячейка С. За счет этого можно видеть падение наполнения ячейки В.

\addimghere{sec4/res1.png}{1}{Результаты запуска модели разделения одной ячейки на две}{img:sec4_res1}

На рисунке \ref{img:sec4_res2} показано, что при заданных начальных параметрах, указанных слева, мы можем видеть что спустя небольшой промежуток времени модель входит в стационарное состояние. Это достигается за счет высокой пропускной способности ячеек B и C и низкой входящей скорости в ячейку A.

\addimghere{sec4/res2.png}{1}{Результаты запуска модели разделения одной ячейки на две}{img:sec4_res2}

\clearpage