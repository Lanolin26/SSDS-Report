\section{Модель дороги из двух ячеек}

\subsection{Описание модели}

На рисунке \ref{img:sec3_model} представлена получившаяся модель системной динамики прямого соединения двух ячеек.

Модель представляет собой два стока (ячейки) и три потока: входящий, промежуточный и выходящий.
Так же присутствует ряд переменных, которые задают поведение ячеек (каждый параметр индивидуален для своей ячейки):
\begin{itemize}
    \item[Speed] - ограничение скорости машины в ячейке;
    \item[V0] - входящая скорость машины;
    \item[N0] - количество машины на 1 км, которые могут находиться в ячейке;
    \item[Ln] - длинна участка ячейки;
    \item[C0] - пропускная способность ячейки на 1 км;
    \item[s 0] - количество машин, которые могут въехать с предыдущей ячейки;
    \item[r 0] - количество машин, которое может выехать из ячейки и попасть в следующую;
\end{itemize}

\addimghere{sec3/model.png}{1}{Модель прямого соединения двух ячеек}{img:sec3_model}

\subsection{Описание результатов}

На рисунке \ref{img:sec3_res1} показано, что при заданных начальных параметрах, указанных слева, мы можем наблюдать, что максимальной пропускной способности выходящего потока не хватает для обеспечения плавного заполнения ячеек превосходящим входным потоком.

\addimghere{sec3/res1.jpg}{1}{Результаты запуска модели}{img:sec3_res1}

На рисунке \ref{img:sec3_res2} показано, что при заданных начальных параметрах, указанных слева, мы можем наблюдать отрицательные значения входящих потоков. Это связано с тем, что мы ограничили скорость на ячейке B сильно ниже, чем на ячейке А. В следствии этого машины, вышедшие из ячейки A, не могут попасть в ячейку B и возвращаются в ячейку А, так как промежуточного состояние нет. Именно такую ситуацию иллюстрируют отрицательные значения входящих потоков, то есть поток машин идет в обратную сторону.

\addimghere{sec3/res2.jpg}{1}{Результаты запуска модели}{img:sec3_res2}

На рисунке \ref{img:sec3_res3} показано, что при заданных начальных параметрах, указанных слева, мы можем наблюдать, что в начале пропускная способность входящего в ячейку А и промежуточного потока A->B достигла своего возможного максимума, но в последствии уменьшилась в связи с максимальным наполнением ячеек.

\addimghere{sec3/res3.jpg}{1}{Результаты запуска модели}{img:sec3_res3}

\clearpage