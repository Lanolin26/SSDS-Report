\section{Модель соединения дорог}

\subsection{Описание модели}

На рисунке \ref{img:sec5_model} представлена получившаяся модель системной динамики соединения двух ячеек в одну.

Модель представляет собой три стока (ячейки) и пять потоков: два входящих, два промежуточных и один выходящий.
Так же присутствует ряд переменных, которые задают поведение ячеек (каждый параметр индивидуален для своей ячейки):
\begin{itemize}
    \item[Speed] - ограничение скорости машины в ячейке;
    \item[V0] - входящая скорость машины;
    \item[N0] - количество машины на 1 км, которые могут находиться в ячейке;
    \item[Ln] - длинна участка ячейки;
    \item[C0] - пропускная способность ячейки на 1 км;
    \item[s 0] - количество машин, которые могут въехать с предыдущей ячейки;
    \item[r 0] - количество машин, которое может выехать из ячейки и попасть в следующую;
\end{itemize}

\addimghere{sec5/model.png}{1}{Модель соединения двух ячеек в одну}{img:sec5_model}

\subsection{Описание результатов}

На рисунке \ref{img:sec5_res1} показано, что при заданных начальных параметрах, указанных слева, мы можем наблюдать, 

\addimghere{sec5/res1.jpg}{1}{Результаты запуска модели}{img:sec5_res1}

На рисунке \ref{img:sec5_res2} показано, что при заданных начальных параметрах, указанных слева, мы можем наблюдать что модель входит в стационарное состояние за счет того, что максимальная пропускная способность ячейки С больше чем выходящие потоки из А и B.

\addimghere{sec5/res2.jpg}{1}{Результаты запуска модели}{img:sec5_res2}

На рисунке \ref{img:sec5_res3} показано, что при заданных начальных параметрах, указанных слева, мы можем наблюдать .

\addimghere{sec5/res3.jpg}{1}{Результаты запуска модели}{img:sec5_res3}

\clearpage