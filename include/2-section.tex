\section{Модель простой ячейки дороги}

\subsection{Описание модели}

На рисунке \ref{img:sec2_model} представлена получившаяся модель системной динамики одного участка дороги. Листинг модели в формате xmile представлен в приложении \hyperlink{app-a}{А}.

Модель представляет собой один сток (ячейка) и два потока: входящий и выходящий.
Так же присутствует ряд переменных, которые задают поведение ячейки:
\begin{itemize}
    \item[Speed A] - ограничение скорости машины в ячейке;
    \item[V0 A] - входящая скорость машины;
    \item[N0] - количество машины на 1 км, которые могут находиться в ячейке;
    \item[Ln A] - длинна участка ячейки;
    \item[C0 A] - пропускная способность ячейки на 1 км;
    \item[s A 0] - количество машин, которые могут въехать с предыдущей ячейки;
    \item[r A 0] - количество машин, которое может выехать из ячейки и попасть в следующую;
\end{itemize}

\addimghere{sec2/model.jpg}{0.8}{Модель одной ячейки дороги}{img:sec2_model}

\subsection{Описание результатов запусков}

На рисунке \ref{img:sec2_res1} показано, что при заданных начальных параметрах, указанных слева, получаем насыщение ячейки машинами и в следствии входящий поток уменьшается практически до нуля. При этом ячейка достигает своего максимального количества машин и не увеличивается бесконечно. Поток \textit{out} медленно уменьшает количество машин, но на их место сразу же поступают машины из потока \textit{in}.

\addimghere{sec2/res1.jpg}{1}{Результаты запуска модели одной ячейки дороги}{img:sec2_res1}

На рисунке \ref{img:sec2_res2} показано, что при заданных начальных параметрах, указанных слева, входящее и выходящее количество машин равно, а пропускная способность много больше максимального количества машин, в следствии чего получаем неменяющееся количество машин в ячейке.

\addimghere{sec2/res2.jpg}{1}{Результаты запуска модели одной ячейки дороги}{img:sec2_res2}

На рисунке \ref{img:sec2_res3} показано, что при заданных начальных параметрах, указанных слева, входящее и выходящее количество машин равно и пропускная способность много больше максимального количества машин, но мы не можем поместить двойной входящий поток в ячейку, поэтому она заполняется до возможного и опустошается, при этом еще дополняется на дельту $N - 2 * s A 0$.

\addimghere{sec2/res3.jpg}{1}{Результаты запуска модели одной ячейки дороги}{img:sec2_res3}

\clearpage